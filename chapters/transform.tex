% !TEX root = ../waves.tex
In the previous chapter, we demonstrated that under good conditions, periodic functions
can be written as an infinite series of elementary trigonometric functions, the Fourier
series. This expansion is crucial for studying oscillatory phenomena, as it decomposes any
oscillating function into a sum of pure sine and cosine waves. Another crucial property
presented in~\cref{eq:cn-der} is that the Fourier coefficients of the derivative of a
function $f$ are simply the coefficients of $f$ multiplied by a power of the frequency.
This property is quite powerful as it can transform linear differential equations into
simpler polynomial equations. However, the Fourier series expansion is in principle
limited to describing periodic functions, and one can legitimately wonder if it can be
generalised to non-periodic functions.

A first interesting observation is that Fourier series can be used to represent
non-periodic functions on any finite interval. Indeed, let us consider a function $f$ on
$\R$, which we consider smooth for the sake of simplicity. Let $A$ be a positive real
number, then according to~\cref{thm:fourier-pt}, for all $x\in(-A,A)$,
\begin{equation}
  f(x)=\sumz{n}c_n\,e^{\frac{\pi}{A}inx}\,,\label{eq:series-nonper}
\end{equation}
with the Fourier coefficients
\begin{equation}
  c_n=\frac{1}{2A}\int_{-A}^{A}\diff x\,f(x)\,e^{-\frac{\pi}{A}inx}\,.\label{eq:cn-nonper}
\end{equation}
This happens because $f$ restricted to $(-A,A)$ can be interpreted as one period of a
$2A$-periodic function $\bar{f}$ defined by $\bar{f}=f$ on $(-A,A)$. Outside $(-A,A)$, the
series~\cref{eq:series-nonper} converges to a periodic copy of $f$ in $(-A,A)$.
Interestingly, this can be done for arbitrarily large $A$, and one can question whether
the whole function $f$ on $\R$ can be described in that way by taking the $A\to+\infty$
limit. We can try to formally conjecture what such a limit would look like, ignoring for
the moment potential convergence issues. We start by recalling the rectangle approximation
for an integral
\begin{equation}
  \lim_{a\to0}\,\sumz{n}a\,F(an)=\int_{-\infty}^{+\infty}\diff x\, F(x)\,,
\end{equation}
where $F$ is an integrable function on $\R$. Using this formula with $a=\frac{1}{2A}$, one
can conjecture that the $A\to+\infty$ limit of~\cref{eq:series-nonper} leads to
\begin{equation}
  f(x)=\int_{-\infty}^{+\infty}\diff\omega\,\hat{f}(\omega)\,e^{2\pi i\omega x},
  \qquad\text{with}\qquad
  \hat{f}(\omega)=\int_{-\infty}^{+\infty}\diff x\,f(x)\,e^{-2\pi i\omega x}\,.
\end{equation}
So on an infinite range, the Fourier coefficient index $n$ becomes a continuous variable.
The associated function $\hat{f}$ is called the \emph{Fourier transform} of $f$, and $f$
and $\hat{f}$ have an interesting duality through the relationships above. This duality is
absolutely fundamental both in physics and mathematics, and can be seen as a
generalisation of Fourier series. Clearly, many things can go wrong when taking the
$A\to+\infty$ limit above, and one needs to derive the formulas above more carefully. It
can be shown that the Fourier transform can be defined for a remarkably wide range of
mathematical objects. However, its proper construction relies on quite technical in-depth
knowledge of functional analysis, going beyond the scope of this introductory course.
Therefore, this chapter is mostly aimed at explaining how to use Fourier transforms
practically, and most results will be admitted.

We start by defining the Fourier transform and studying its properties for a simple class
of functions for which the integrals above are clearly defined. In all this chapter, we
will directly give results in $d$ dimensions.
%%%%%%%%%%%%%%%%%%%%%%%%%%%%%%%%%%%%%%%%%%%%%%%%%%%%%%%%%%%%%%%%%%%%%%%%%%%%%%%%%%%%%%%%%%
\section{Fourier transform of Schwartz functions}
%-----------------------------------------------------------------------------------------
\subsection{Definition and properties}
\begin{definition}
  We call \emph{multi-index} $\alpha$ in $d$ dimensions any $d$-tuple
  $\alpha=(\alpha_1,\dots,\alpha_d)$ of  non-negative integers. We note
  $|\alpha|=\alpha_1+\cdots+\alpha_d$ the sum of all components of $\alpha$.
  For any vector $\vec{x}\in\R^d$, we also define the power
  \begin{equation}
    \vec{x}^\alpha=x_1^{\alpha_1}\cdots x_d^{\alpha_d}\,,
  \end{equation}
  and similarly
  \begin{equation}
    \partial^{\alpha}=\partial_1^{\alpha_1}\cdots\partial_d^{\alpha_d}\,,
  \end{equation}
  where $\partial_j$ is the partial derivative in the $j$-th variable.
\end{definition}
\begin{example}
  We consider in $3$ dimensions the multi-index $\alpha=(2,0,1)$. Let $f$ be a
  differentiable function on $\R^3$ and $\vec{x}=(x,y,z)$. Then
  \begin{equation}
    \vec{x}^\alpha=x^2z\qquad\text{and}\qquad\partial^\alpha
    f(\vec{x})=\frac{\partial^3}{\partial x^2\partial z}\,f(x,y,z)\,.
  \end{equation}
\end{example}
\begin{definition}
  \label{def:schwartz-fn}
  A function $f$ on $\R^d$ is called a \emph{Schwartz function} if it is infinitely
  differentiable, and the function and all its derivatives decay faster than any power
  at infinity. More explicitly, for all pairs of multi-indices $\alpha$ and $\beta$,
  \begin{equation}
    \partial^{\alpha}f\quad\text{exists everywhere,}\qquad\text{and}\qquad
    \lim_{|\vec{x}|\to+\infty}\vec{x}^\beta\partial^{\alpha}f(\vec{x})=0\,.
  \end{equation}
\end{definition}
\begin{definition}
  We call \emph{Gaussian kernel} with \emph{width} $\sigma$ the function $\gauss_{\sigma}$
  defined on $\R^d$ by
  \begin{equation}
    \gauss_{\sigma}(\vec{x})=\frac{1}{(2\pi)^{\frac{d}{2}}\sigma^d}\,e^{-\frac{|\vec{x}|^2}{2\sigma^2}}\,,
  \end{equation}
  where $\sigma$ is a positive real number.
\end{definition}
\begin{proposition}
  The Gaussian kernel $\gauss_{\sigma}$ is a Schwartz function for any width $\sigma$.
\end{proposition}
\begin{example}
  The function $f$ defined for $\vec{x}\in\R^3$ by
  \begin{equation}
    f(\vec{x})=\frac{1}{|\vec{x}|^2+2}\,,
  \end{equation}
  is not a Schwartz function, although it is infinitely differentiable. Indeed,
  \begin{equation}
    \lim_{x_1\to+\infty} x_1^2\,f(\vec{x})=1\,,
  \end{equation}
  which does not satisfy~\cref{def:schwartz-fn} for the multi-indices $\alpha=(0,0,0)$ and
  $\beta=(2,0,0)$.
\end{example}
\begin{example}
  The function $f$ defined for $x\in\R$ by
  \begin{equation}
    f(x)=e^{-|x|}\,,
  \end{equation}
  is not a Schwartz function, although it decays faster than any powers of $x$. Indeed,
  \begin{equation}
    f'(x)=-\sign(x)\,e^{-|x|}\,,
  \end{equation}
  is discontinuous at $0$, and therefore $f$ does not admit a second derivative.
\end{example}
\begin{proposition}
  An arbitrary linear combination of Schwartz functions is a Schwartz function. A product
  of Schwartz functions is also a Schwartz function.
\end{proposition}
\begin{definition}
  Let $f$ be a Schwartz function on $\R^d$, the function $\hat{f}$ defined on $\R^d$ by
  \begin{equation}
    \hat{f}(\vec{k})=\int_{\R^d}\diff^d\vec{x}\,f(\vec{x})\,e^{-i\vec{k}\cdotp\vec{x}}\,,
  \end{equation}
  is called the \emph{Fourier transform} of $f$. We additionally note $\mathcal{F}$ the operator
  that associate $f$ to its Fourier transform $\mathcal{F}f=\hat{f}$.
\end{definition}
% \begin{proposition}
%   Let $f$ be a Schwartz function on $\R^d$, the following properties hold
%   \begin{enumerate}
%     \item Conjugation: $\hat{f}^*=\mathcal{R}\mathcal{F}(f^*)$
%     \item
%   \end{enumerate}
% \end{proposition}
%-----------------------------------------------------------------------------------------
\subsection{Inverse Fourier transform}
%-----------------------------------------------------------------------------------------
\subsection{Convolution product}
%%%%%%%%%%%%%%%%%%%%%%%%%%%%%%%%%%%%%%%%%%%%%%%%%%%%%%%%%%%%%%%%%%%%%%%%%%%%%%%%%%%%%%%%%%
\section{Generalised functions -- \textit{There be dragons}}
%%%%%%%%%%%%%%%%%%%%%%%%%%%%%%%%%%%%%%%%%%%%%%%%%%%%%%%%%%%%%%%%%%%%%%%%%%%%%%%%%%%%%%%%%%
\section{Back to Fourier series}
%%%%%%%%%%%%%%%%%%%%%%%%%%%%%%%%%%%%%%%%%%%%%%%%%%%%%%%%%%%%%%%%%%%%%%%%%%%%%%%%%%%%%%%%%%
\section{Inversion of differential operators}