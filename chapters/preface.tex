% !TEX root = ../waves.tex
We all have an intuitive notion that vibrating phenomena can be decomposed into modes of
different frequencies. For example, we know that music is primarily composed by
superimposing sounds with different pitches; we can observe waves mixing on the surface of
liquids, and we understand that white light can be decomposed into several colours when
observing rainbows in the atmosphere. Scientists from antiquity to modern times have
utilised this notion to describe periodic phenomena. However, it was not until the early
19\textsuperscript{th} century that the first general mathematical description of such
decomposition was formulated, through the transformative paper by Joseph Fourier,
``Mémoire sur la propagation de la chaleur dans les corps solides''\footnote{Translation:
``Treatise on the propagation of heat in solid bodies''.}. In this work, Fourier
demonstrated for the first time that periodic functions can be generally expressed as a
superposition of sine and cosine waves of different frequencies, thus initiating the field
known today as \emph{Fourier Analysis}. This description quickly proved to be fundamental
to understanding any vibrating phenomenon in physics and is now a key mathematical concept
in a wide range of applications, including engineering, music and video production, signal
processing, quantum and classical physics, pure mathematics, finance, and many others. One
can therefore confidently state that Fourier Analysis is one of the most crucial aspects
of modern mathematics, and its knowledge is essential for anyone wishing to mathematically
describe virtually any dynamic phenomenon.

This course is intended to provide a first introduction to waves and Fourier analysis for
undergraduate students in mathematical physics. It assumes the reader has a basic
knowledge of calculus of functions of one or several real variables, trigonometric
functions, complex numbers, and vector calculus. On the physics side, it additionally
assumes knowledge of classical mechanics and Newton's laws of motion. The outline of the
course mainly follows the chronological development of Fourier analysis.
\cref{chap:wave-eq} is an introduction to the wave equation in the context of vibrating
strings, which is one of the physical problems that historically motivated the development
of Fourier analysis. In~\cref{chap:elementary-waves}, elementary notions on periodic
functions, and sine and cosine waves are introduced. In~\cref{chap:series}, the Fourier
expansion of a periodic function into a series of sine and cosine waves is formally
described. In~\cref{chap:transform}, the Fourier transform is introduced, generalising the
concept of Fourier expansion to non-periodic functions.