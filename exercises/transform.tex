% !TEX root = ../waves.tex
%%%%%%%%%%%%%%%%%%%%%%%%%%%%%%%%%%%%%%%%%%%%%%%%%%%%%%%%%%%%%%%%%%%%%%%%%%%%%%%%%%%%%%%%%%
\section{Exercises}
\begin{ExerciseList}
  %---------------------------------------------------------------------------------------
  \Exercise[label=ft-trans] Prove~\cref{prop:ft-trans}. For the \emph{differentiation in frequency}
  property, we assume we can use the \emph{Leibniz integral rule}, which can be formulated as follows.
  Let $F(x,\omega)$ be a function of two real variables such that $|F(x,\omega)|$ is integrable,
  $|\smash{\pd{}{\omega}F(x,\omega)}|$ exists everywhere and is integrable, then the function defined
  by
  \begin{equation}
    \phi(\omega)=\intr{x}F(x,\omega)\,,
  \end{equation}
  is differentiable and,
  \begin{equation}
    \phi'(\omega)=\intr{x}\pd{}{\omega}F(x,\omega)\,.
  \end{equation}
  %---------------------------------------------------------------------------------------
  \Exercise[label=gauss-schwartz] This exercise is a guided proof
  of~\cref{prop:gauss-schwartz}, as well as an introduction
  to Hermite polynomials.
  \Question Let $\sigma$ be a positive real number. Prove that for all positive
  integer $\alpha$,
  \begin{equation}
    \gauss_{\sigma}^{(\alpha)}=P_{\alpha}\gauss_{\sigma}\,,
  \end{equation}
  where $\gauss_{\sigma}$ is the Gaussian kernel introduced in~\cref{def:gauss},
  and $P_{\alpha}$ is a degree $\alpha$ polynomial verifying the recurrence relation
  \begin{equation}
    P_{\alpha+1}=P_{\alpha}'+P_1P_{\alpha}\,.
  \end{equation}
  \Question Prove that $\gauss_{\sigma}$ is a Schwartz function (\cref{prop:gauss-schwartz}).
  \Question The $n$-th \emph{Hermite polynomial} $H_n$ is defined by
  \begin{equation}
    H_n(x)=(-1)^ne^{x^2}\frac{\diff^n}{\diff x^n}(e^{-x^2})\,.
  \end{equation}
  Show that
  \begin{equation}
    H_n=(-1)^nP_n\qquad\text{with}\qquad \sigma=\frac{1}{\sqrt{2}}\,.
  \end{equation}
  \Question Compute explicitly $H_1$, $H_2$, and $H_3$.
  \Question Show that the Hermite polynomials are orthogonal for the dot product
  \begin{equation}
    \braket{f,g}=\intr{x}f(x)g(x)\,e^{-x^2}\,,
  \end{equation}
  \ie that $\braket{H_n,H_m}=0$ if $n\neq m$.
  %---------------------------------------------------------------------------------------
  \Exercise[label=gauss-uncertainty]
  This exercise is a guided proof of the
  Gaussian uncertainty principle (\cref{thm:gauss-uncertainty}).
  \Question Show that
  \begin{equation}
    \intr{x}e^{-\pi x^2}=1\,.
  \end{equation}
  \emph{Hint: consider the identity }
  $(\intr{x}e^{-\pi x^2})^2=\intr{x}\intr{y}e^{-\pi (x^2+y^2)}$.
  \Question Show that
  \begin{equation}
    \intr{x}\gauss_{\sigma}(x)=1\,,
  \end{equation}
  for all widths $\sigma>0$.
  \Question Show that the Fourier transform of the Gaussian kernel $\hat{\gauss}_{\sigma}$
  is a solution of the differential equation
  \begin{equation}
    \hat{\gauss}_{\sigma}'(\omega)=-4\pi^2\sigma^2\omega\,\hat{\gauss}_{\sigma}(\omega)\,,
  \end{equation}
  with the initial condition $\hat{\gauss}_{\sigma}(0)=1$.
  \Question By solving the equation above, prove~\cref{thm:gauss-uncertainty}.
  %---------------------------------------------------------------------------------------
  \Exercise[label=exp-decay]
  We consider the function $f$ defined by
  \begin{equation}
    f(t)=\frac{e^{-m|t|}}{2m}\,,
  \end{equation}
  for $t\in\R$ and where $m$ is a positive real number. Its Fourier transform is given by
  \begin{equation}
    \hat{f}(\omega)=\frac{1}{4\pi^2\omega^2+m^2}\,.
    \label{eq:expdecay-ft}
  \end{equation}
  In this exercise we derive the formula above in two different ways.
  \Question Compute directly \cref{eq:expdecay-ft} through the Fourier transform integral.
  \Question Show that, in the sense of distributions,
  \begin{equation}
    -f''(t)+m^2f(t)=\delta(t)\,,
  \end{equation}
  and deduce \cref{eq:expdecay-ft} from this equation.
  \InputIfFileExists{solutions/transform}{}{}
\end{ExerciseList}