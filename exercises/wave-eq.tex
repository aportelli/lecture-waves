% !TEX root = ../waves.tex
%%%%%%%%%%%%%%%%%%%%%%%%%%%%%%%%%%%%%%%%%%%%%%%%%%%%%%%%%%%%%%%%%%%%%%%%%%%%%%%%%%%%%%%%%%
\section{Exercises}
\begin{ExerciseList}
  %---------------------------------------------------------------------------------------
  \Exercise[label=odesincos] Let $f$ be a twice-differentiable function on $\mathbb{R}$ which is
  a solution of the ordinary differential equation
  \begin{equation}
    f''(t) + A^2 f(t) = 0\,,
  \end{equation}
  where $A$ is a non-zero real number. We additionally define the functions $g$ and $h$
  with
  \begin{align}
    g(t)&=f(t)\cos(At)-\frac{f'(t)}{A}\sin(At)\label{eq:csgdef}\,,\\
    h(t)&=f(t)\sin(At)+\frac{f'(t)}{A}\cos(At)\label{eq:cshdef}\,.
  \end{align}
  \Question Show that $g$ and $h$ are constant functions.
  \Question Deduce from the previous question that there exists two real numbers $a$ and $b$
  such that
  \begin{equation}
    f(t)=a\cos(At)+b\sin(At)\,.\label{eq:csfsol}
  \end{equation}
  %---------------------------------------------------------------------------------------
  \Exercise[label=stringhooke] We start by recalling Hooke's law for the restoring force
  of a stretched or compressed spring. We assume a spring is aligned with the $x$-axis and
  has an equilibrium length of $\ell$. Hooke's law states that if the spring is stretched
  or compressed to a length $\ell+\delta\ell$, the tension forces at the extremities of
  the spring have a magnitude proportional to the displacement:
  \begin{equation}
    T=k|\delta\ell|\,,
  \end{equation}
  where $k$ is the \emph{spring constant} in $\newton\per\metre$. Additionally, if the
  spring is compressed, the forces are always directed towards the equilibrium position,
  \ie outward when the spring is compressed and inward when it is stretched. \Question
  From the point of view of Hooke's law, argue that the tension forces on the discretised
  string in~\cref{eq:stringtm,eq:stringtp} could be described in a more accurate way.
  \Question Use Hooke's law to generalise the equations of
  motion~\cref{eq:string-fullxeq,eq:string-fullyeq}.
  \Question Demonstrate that, assuming longitudinal movement is negligible, the
  generalisations above do not change the wave equation~\cref{eq:wave-eq} in the limit of
  small amplitudes.
  %---------------------------------------------------------------------------------------
  \Exercise[label=guitar] We consider a guitar string with a vibrating length
  $L=\unit{63}{\centi\metre}$ and a linear mass density $\mu=\unit{1.2}{\gram\per\metre}$.
  \Question What tension is required to apply to the string to tune it to the frequency
  $\unit{110}{\hertz}$? \Question Assuming the same tension as in Question 1, what is the
  wave speed on the string? \Question A finger is placed on the string at a distance
  $\ell$ from the nut (the extremity of the string farthest from the body), how is the
  string vibration frequency modified? \Question Assuming the same tension as in Question
  1, what finger position leads to a frequency of $\unit{164.81}{\hertz}$? \Question Frets
  on the fingerboard are spaced such that the frequency of the string increases by one
  halftone between two frets. A frequency $\omega'$ is one halftone higher than $\omega$
  if
  \begin{equation}
    \omega'=2^{\frac{1}{12}}\omega\,.
  \end{equation}
  Compute the distance from the nut of the first few frets.
  %---------------------------------------------------------------------------------------
  \InputIfFileExists{solutions/wave-eq}{}{}
\end{ExerciseList}
