% !TEX root = ../waves.tex
%%%%%%%%%%%%%%%%%%%%%%%%%%%%%%%%%%%%%%%%%%%%%%%%%%%%%%%%%%%%%%%%%%%%%%%%%%%%%%%%%%%%%%%%%%
\section{Exercises}
\begin{ExerciseList}
  %---------------------------------------------------------------------------------------
  \Exercise[label=chebyshev]
  For any real number $\theta$ and any integer $n\leq 0$, we consider the function $T_n$ defined by
  \begin{equation}
    T_n[\cos(\theta)]=\cos(n\theta)\,.
  \end{equation}
  $T_n$ is called the $n$-th \emph{Chebyshev polynomial}. It is not clear from the formula
  above that $T_n$ is in fact a polynomial, and the aim of this exercise is to prove that fact.
  \Question Show that $T_0(x)=1$, $T_1(x)=x$, and $T_2(x)=2x^2-1$.
  \Question Show that for all $n\geq 1$
  \begin{equation}
    T_{n+1}(x)=2x T_n(x)-T_{n-1}(x)\,,
  \end{equation}
  and deduce that $T_n(x)$ is a polynomial in $x$.
  \Question Compute explicitly $T_4(x)$.
  %---------------------------------------------------------------------------------------
  \Exercise[label=viete]~
  \Question Derive the double-angle formula
  \begin{equation}
    \sin(x)=2\sin\left(\frac{x}{2}\right)\cos\left(\frac{x}{2}\right)\,,
  \end{equation}
  where $x\in\mathbb{R}$.
  \Question Show by induction that for all integers $n\geq 1$,
  \begin{equation}
    \sin(x)=2^n\sin\left(\frac{x}{2^n}\right)\left[\prod_{j=1}^n\cos\left(\frac{x}{2^j}\right)\right]\,.
  \end{equation}
  \Question Deduce, for $x\neq 0$, \emph{Viète's infinite product}
  \begin{equation}
    \frac{\sin(x)}{x}=\prod_{n=1}^{+\infty}\cos\left(\frac{x}{2^n}\right)\,,
  \end{equation}
  \Question Finally, derive \emph{Viète's formula}
  \begin{equation}
    \frac{2}{\pi}=\frac{\sqrt{2}}{2}\frac{\sqrt{2+\sqrt{2}}}{2}\frac{\sqrt{2+\sqrt{2+\sqrt{2}}
    }}{2}\cdots\,.
  \end{equation}
  %---------------------------------------------------------------------------------------
  \Exercise[label=ampphase]
  In this chapter we defined the sine-cosine and complex coefficients of a trigonometric polynomial,
  in~\cref{eq:trigp-sc,eq:trigp-complex}, respectively. In this exercise we derive an additional form, generally called the \emph{amplitude-phase} form.
  \Question Let $a$, $b$, and $\omega$ be three real numbers such that $a^2+b^2\neq0$. Show that for all $t\in\mathbb{R}$
  \begin{equation}
    a\cw(t;\omega)+b\sw(t;\omega)=A\cw(t;\omega,-\phi)\,,
  \end{equation}
  where $A=\sqrt{a^2+b^2}$, and $\phi$ is such that
  \begin{equation}
    \cos(\phi)=\frac{a}{\sqrt{a^2+b^2}}\qquad\text{and}\qquad
    \sin(\phi)=\frac{b}{\sqrt{a^2+b^2}}\,.
  \end{equation}
  \Question Let $P$ be a real trigonometric polynomial of degree $N$. Show that $P$ can be written
  \begin{equation}
    P(t)=\frac{A_0}{2}+\sum_{n=1}^NA_n\cw(t;n\omega,-\phi_n)\,,
  \end{equation}
  and express $A_n$ and $\phi_n$ in function of $a_n$, $b_n$, and $c_n$. $A_n$ and $\phi_n$ are called the \emph{amplitude-phase} coefficients of $P$.
  \Question Discuss the unicity of the amplitude-phase coefficients.
  %---------------------------------------------------------------------------------------
  \InputIfFileExists{solutions/elementary-waves}{}{}
\end{ExerciseList}